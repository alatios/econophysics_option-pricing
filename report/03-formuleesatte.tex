\chapter{(C) Formule di valutazione esatte} \label{cap:exactformulas}

In questo capitolo viene valutata la formula del pay-off dell'opzione \textit{performance corridor} (\ref{eq:performancecorridor_barrier}) in diversi casi limite. 

\section{(C-0) Formule analitiche per i limiti asintotici della barriera}

La prima variabile che viene presa in considerazione nella formula (\ref{eq:performancecorridor_barrier}) è $B$, cioè una variabile adimensionale che indica la grandezza della barriera. Si studiano i seguenti limiti:
\begin{itemize}
    \item $B=0$
    
    Nel caso $B=0$ la condizione $\left| \frac{1}{\Delta t} \ln{\frac{S(t_{i+1})}{S(t_i)}} \right| < B \cdot \sigma$ non è mai rispettata e la quantità $P_{i}$ assume sempre un valore nullo. In questo modo il pay-off risulta sempre nullo poiché la quantità $[-K]^{+}$ vale $0$ essendo K una costante positiva nell'intervallo $[0,1]$:
    \begin{equation}
        F_{pc} = 0
    \end{equation}
    
    Le simulazioni effettuate con $B=0$ restituiscono un prezzo dell'opzione nullo confermando così il risultato previsto. Inoltre il prezzo ottenuto è compatibile con 0 fino a valori di $B=0.2$.
    
    \item $B \rightarrow \infty$
    
    Nel caso $B \rightarrow \infty$ la condizione $\left| \frac{1}{\Delta t} \ln{\frac{S(t_{i+1})}{S(t_i)}} \right| < B \cdot \sigma$ è sempre rispettata e la quantità $P_{i}$ assume sempre un valore pari a $1$. In questo modo il pay-off risulta sempre essere:
    \begin{equation}
        F_{pc}=N[1-K]
    \end{equation}
    dove la parte positiva è stata tralasciata poiché $K \in [0,1]$. Si noti che quest'ultima espressione dovrà essere attualizzata al valore della data odierna moltiplicando il pay-off per il fattore $e^{r T}$.
    
    Nella tabella \ref{tab:Teo-MC_B} è riportato il confronto tra i valori ottenuti dalle simulazioni e quelli analitici previsti per diversi valori di $K$ prendendo in considerazione:
\begin{itemize}
    \item Numero di simulazioni totali = $100 \times 10^6$;
    \item Prezzo iniziale = 100 [USD];
    \item Prezzo di esercizio $E$= 100 [USD];
    \item Data di maturità $T$ = 1 anno; 
    \item Numero di intervalli in cui è divisa la data di maturità = 400;
    \item Tasso di interesse privo di rischio $r$ = $0.01\%$;
    \item Volatilità $\sigma$ = $25\%$;
    \item Nozionale N = 1 [USD].
\end{itemize}
    
    \begin{table}[t]
\small
\centering
\begin{tabular}{|l|l|l|l|l|l|l|l|}
\hline
$B$ & $K$ & $F_{MC}$ & $\epsilon_{MC}$ & $F_{forecast}$ & Distanza in unità di $\epsilon_{MC}$ \\
\hline \hline
$100$ & $0.3$ & $0.6999299490$ & $1.8 \cdot 10^{-7}$ &
$0.6999300035$ & $0.3$\\ \hline
$50$ & $0.5$ & $0.4999499621$ & $1.4 \cdot 10^{-7}$ & $0.4999500025$ & $0.29$\\ \hline
$10$ & $0.6$ & $0.3999599664$ & $1.2 \cdot 10^{-7}$ & $0.399960002$ & $0.29$\\ \hline
$5$ & $0.2$ & $0.7999193802$ & $1.9 \cdot 10^{-7}$ & $0.799920004$ & $3.3$\\ \hline
\end{tabular}
\caption{Confronto tra i risultati delle simulazioni Monte Carlo e quelli previsti per $B \rightarrow \infty$.}
\label{tab:Teo-MC_B}
\end{table}
\end{itemize}

Si noti che i risultati ottenuti attraverso le simulazioni sono compatibili con i valori previsti poiché il prezzo dell'opzione ottenuto attraverso le simulazioni Monte Carlo è contenuto nell'intervallo $[F_{forecast}-\epsilon_{MC},F_{forecast}+\epsilon_{MC}]$ per valori di $B$ maggiori di 10. Nei casi in cui $B<10$ la formula ricavata precedentemente non è più applicabile e i risultati delle simulazioni non sono compatibili con quelli previsti da essa.

\section{(C-1) Formula analitica per singola data di rilevazione}

In questa parta viene presa in considerazione la variabile $m$ che indica le date di rivelazione ricavando la formula analitica esatta nel caso di $m=1$. Ponendo quesa condizione di ottine:
\begin{equation}
    F_{pc}=N[P-K]^{+}
\end{equation}
dove:
\begin{equation}
    P_i = \begin{cases}
    1, & \text{se} \,\,\left| \ln{\frac{S(t_{i+1})}{S(t_i)}} \right| < B \sigma \sqrt{\Delta t};\\
    0, & \text{altrimenti}.
    \label{eq:pc_condition}
  \end{cases}
\end{equation}
che si riscrive nella forma:
\begin{equation}
    P_i = \begin{cases}
    N[1-K], & \text{se} \,\,\left| \ln{\frac{S(t_{i+1})}{S(t_i)}} \right| < B \sigma \sqrt{\Delta t};\\
    0, & \text{altrimenti}.
  \end{cases}
\end{equation}
dove è stata omessa la parte intera.

Utilizzando il modello del $moto browniano geometrico$ secondo il quale il prezzo del sottostante segua un processo lognormale, per il lemma di Ito il valore che il prezzo assume alla rilevazione $i+1$ considerando un'unica data di $fixing$ è dato dalla formula:
\begin{equation}
    S(t_{i+1}) = S(t_i) \exp{\left[\left(r- \frac{\sigma^2}{2}\right)T + \sigma \sqrt{T} w\right]}.
\end{equation}
pertanto il pay-off assume il valore $N[1-K]$ se e solo se viene rispettata la condizione:
\begin{equation}
    \left| \left(r- \frac{\sigma^2}{2}\right) \frac{\sqrt{T}}{\sigma} + \omega \right| < B;
    \label{eq:condition}
\end{equation}

Il prezzo dell'opzione sarà quindi dato dal valore del suo pay-off nel caso in cui questo non sia nullo, moltiplicato per la probabilità che tale scenario si avveri e per il fattore di attualizzazione;
\begin{equation}
    F_{pc}=N[1-K] \cdot P(P=1) \cdot e^{-rT}
\end{equation}
La probabilità $P(P=1)$ rappresenta la probabilità che la condizione \ref{eq:condition} sia rispettata. Essendo $\omega$ un numero casuale gaussiano con media nulla e varianza unitaria si può ottenere:
\begin{equation}
    P(P=1) = N(B-c) - N(-B-c)
\end{equation}
dove:
\begin{equation}
    c=\left(r- \frac{\sigma^2}{2}\right) \frac{\sqrt{T}}{\sigma}
\end{equation}
e $N(x)$ indica la funzione di distribuzione normale cumulata (\ref{eq:cumulative_gaussian})

In conclusione la formula analitica esatta dell'opzione per una singola data di $fixing$ è data dalla seguente formula:
\begin{equation}
    F_{pc}=N[1-K] \cdot \left[N(B-c) - N(-B-c) \right] \cdot e^{-rT}
\end{equation}


Nella tabella \ref{tab:Teo-MC_m} è riportato il confronto tra i valori ottenuti dalle simulazioni e quelli analitici previsti per diversi valori di $B$, $K$ e $\sigma$ prendendo in considerazione:
\begin{itemize}
    \item Numero di simulazioni totali = $100 \times 10^6$;
    \item Prezzo iniziale = 100 [USD];
    \item Prezzo di esercizio $E$= 100 [USD];
    \item Data di maturità $T$ = 1 anno; 
    \item Numero di intervalli in cui è divisa la data di maturità = 1;
    \item Tasso di interesse privo di rischio $r$ = $0.01\%$;
    \item Nozionale N = 1 [USD].
\end{itemize}

\begin{table}[t]
\small
\centering
\begin{tabular}{|l|l|l|l|l|l|l|l|}
\hline
$B$ & $K$ & $\sigma$ & $F_{MC}$ & $\epsilon_{MC}$ & $F_{forecast}$ & Distanza in unità di $\epsilon_{MC}$ \\
\hline \hline
$0.25$ & $0$ & $0.3 \%$ & $0.1951816009$ & $3.96316 \cdot 10^{-5}$ & $0.1952396044$ & $1.46$\\ \hline
$0.2$ & $0.1$ & $0.25\%$ & $0.1415115231$ & $3.27601 \cdot 10^{-5}$ & $0.1415647139$ & $1.62$\\ \hline
$0.45$ & $0.2$ & $0.4 \%$ & $0.2726193287$ & $3.79149 \cdot 10^{-5}$ & $0.2726736347$ & $1.43$\\ \hline
$0.5$ & $0.2$ & $0.3\%$ & $0.3031152150$ & $3.88058 \cdot 10^{-5}$ & $0.3031712366$ & $1.44$\\ \hline
$2$ & $0.2$ & $0.25 \%$ & $0.7621755226$ & $1.69611 \cdot 10^{-5}$ & $0.7621806905$ & $0.3$\\ \hline
$5$ & $0.4$ & $0.6 \%$ & $0.5999392531$ & $6.7 \cdot 10^{-8}$ & $0.5999391885$ & $0.96$\\ \hline
\end{tabular}
\caption{Confronto tra i risultati delle simulazioni Monte Carlo e quelli previsti per $m=1$.}
\label{tab:Teo-MC_m}
\end{table}

Si noti che i risultati ottenuti attraverso le simulazioni evidenziano una buona compatibilità con i valori previsti poiché il prezzo dell'opzione ottenuto attraverso le simulazioni Monte Carlo è contenuto nell'intervallo $[F_{forecast}-\epsilon_{MC},F_{forecast}+\epsilon_{MC}]$ per tutti i valori di $B$, $K$ e $\sigma$ considerati.


\section{(C-2) Formula analitica per date di rilevazione asintoticamente infinite}

In questa parta viene preso in considerazione il limite $m \rightarrow \infty$; in questo limite, cioè con il numero di date di rilevazione che tende all'infinito, l'intervallo di tempo che intercorre tra una rilevazione del prezzo del sottostante e la successiva tende a zero, ovvero per $m \rightarrow \infty$ si ottine che $\Delta t \rightarrow 0$. Di conseguenza nella \eqref{eq:exactprice} è possibile trascurare il primo termine in $\Delta t$ ottenendo cosi la formula:
\begin{equation}
     S(t_{i+1}) = S(t_i) \exp{\left[ \sigma \sqrt{\Delta t} w\right]}.
\end{equation}

Si può facilmente riscrivere la condizione \eqref{eq:pc_condition} nel seguente modo:
\begin{equation}
    P_i = \begin{cases}
    1, & \text{se} \,\,\left| \omega \right| < B;\\
    0, & \text{altrimenti}.
    \label{eq:condition_m}
  \end{cases}
\end{equation}

Nel limite considerato la quantità $\left[ \sum_{i=0}^{m-1}{P_i} - K \right]^+$ che compare nella formula del pay-off \eqref{eq:performancecorridor_payoff}, tende alla probabilità che la variabile $P_{i}$ assuma il valore $1$:
\begin{equation}
    \lim_{m \to \infty} \left[ \sum_{i=0}^{m-1}{P_i} - K \right]^+ = P\left(P_{i}=1\right)
\end{equation}

Analogamente al caso precedente, essendo $\omega$ un numero casuale gaussiano con media nulla e varianza unitaria, si può ottenere:
\begin{equation}
    P(P_{i}=1) = N(B) - N(-B)
\end{equation}
dove $N(x)$ indica la funzione di distribuzione normale cumulata (\ref{eq:cumulative_gaussian}).

In conclusione la formula analitica esatta dell'opzione per il limite con un numero elevato di date di $fixing$ è data dalla seguente formula:
\begin{equation}
    F_{pc}=N[N(B) - N(-B)-K]  \cdot e^{-rT}
\end{equation}